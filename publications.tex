%----------------------------------------------------------------------------------------
%	Publications
\section{{\textcolor{companycolor}{\faEditS}}\hspace{0.1cm}Publications}

%----------------------------------------------------------------------------------------
%	BGP Origin Validation (RPKI)
%----------------------------------------------------------------------------------------
\begin{tabularx}{1\linewidth}{>{\raggedleft\scshape}p{2.5cm}X}
\gray SURFnet & \textbf{\href{http://rp.delaat.net/2012-2013/p59/report.pdf}{BGP Origin Validation (RPKI)}} \hfill June 2013\\
\end{tabularx}

\vspace{2pt}
The goal is to increase the adoption rate of RPKI. Hence we've build SURFnet's RPKI Dashboard, which provides network operators with a number of RPKI statistics and assists them in cleaning up possible invalid prefixes.\\ {\faAlternateExternalLink}  \url{https://labs.ripe.net/Members/javy_de_koning/rpki-dashboard} {\faAlternateExternalLink} \url{http://goo.gl/bwOxjD}
\vspace{12pt}

%----------------------------------------------------------------------------------------
%	Defending against DNS reflection amplification attacks
%----------------------------------------------------------------------------------------
\begin{tabularx}{1\linewidth}{>{\raggedleft\scshape}p{2.5cm}X}
\gray NLnet Labs & \textbf{\href{http://www.nlnetlabs.nl/downloads/publications/report-rrl-dekoning-rozekrans.pdf}{Defending against DNS reflection amplification attacks}} \hfill February 2013\\
\end{tabularx}

\vspace{2pt}
The research goal was to find out if the proposed mechanisms to defend against a DNS amplification attacks are effective. The main conclusion is that response rate limiting is a proper defense mechanism against current amplification attacks, but it is not effective against more sophisticated attacks.{\faAlternateExternalLink} \url{http://goo.gl/yWn9vS} 
\vspace{12pt}

%----------------------------------------------------------------------------------------
%	ABN HACK
%----------------------------------------------------------------------------------------
\begin{tabularx}{1\linewidth}{>{\raggedleft\scshape}p{2.5cm}X}
\gray UvA & \textbf{\href{http://staff.science.uva.nl/~delaat/news/2013-02-12/security_in_mobile_banking_ssn.pdf}{Security in mobile banking}} \hfill December 2012\\
\end{tabularx}

\vspace{2pt}
The goal of the research was to find out how the security used in Android based mobile banking applications is implemented. During this project we discovered a serious weakness in the ABN AMRO mobile banking Android app which was exposing PIN and transaction information. {\faAlternateExternalLink} \url{http://goo.gl/P66Xvn} {\faAlternateExternalLink} \url{https://www.security.nl/posting/40061}
\vspace{12pt}